\documentclass{beamer}

\usepackage{listings}
\usepackage{color}

% code definitions
\definecolor{dkgreen}{rgb}{0,0.6,0}
\definecolor{gray}{rgb}{0.5,0.5,0.5}
\definecolor{mauve}{rgb}{0.58,0,0.82}
\lstset{frame=tb,
  language=Python,
  aboveskip=3mm,
  belowskip=3mm,
  showstringspaces=false,
  columns=flexible,
  basicstyle={\small\ttfamily},
  numbers=left,
  numberstyle=\tiny\color{gray},
  keywordstyle=\color{blue},
  commentstyle=\color{dkgreen},
  stringstyle=\color{mauve},
  breaklines=true,
  breakatwhitespace=true
  tabsize=2
}

\title{Practical Types for Python}
\author{Daniel Randall}
\institute{Imperial College London} % (optional)

\date{16th September 2014}

\begin{document}

	\frame{\titlepage}

\begin{frame}
    \frametitle{Introduction}
   	\framesubtitle{Defining the problem}
  % 	Python is a dynamically typed language
   	\begin{block}{Problem}
   		\begin{itemize}
   		  \item Lack of a static compiler means simple type errors are not caught
   		  \item Variables can be a number of different types at a given point
   		\end{itemize}
   	\end{block}

	\begin{block}{What this means}
		\begin{itemize}
		  \item Type errors can be hidden in the source code
		  \item Finding the cause of the type error may take time
		\end{itemize}
   	\end{block}
\end{frame}	
	
  % Frame 1.1 - Examples
\begin{frame}[fragile]
    \frametitle{Introduction}
   	\framesubtitle{Problem examples}
   	\begin{block}{}
   	\begin{lstlisting}
        abs("Hello world")
    \end{lstlisting}
    \end{block}
    \begin{block}{}
    \begin{lstlisting}
        def f(x):
          abs(x)
          x + "Hello world"
    \end{lstlisting}
    \end{block}
\end{frame}
  
  % Frame 1.5 - Current solutions
  \begin{frame}
    \frametitle{Introduction}
   	\framesubtitle{Current tools}
   	The current tools  suffer from the following problems:
   	\begin{itemize}
    		\item Limited type checking
    		\item Require modifications to source code
    		\item Suffer from false positives
    \end{itemize}
  \end{frame}
  
  % Frame 2 - Discuss solution
  \begin{frame}
    \frametitle{Our solution}
    %\framesubtitle{A bit more information about this}
    We attempt to solve this problem using the following techniques: \\
    \begin{itemize}
        \item Modelling the flow of types
    		\item Inferring the types of function arguments
    		\item Taking a cautious approach to \texttt{try}/\texttt{except} control flow
    		\item Distinguishing between global and local variables
    \end{itemize}
    We apply these techniques to a subset of the Python language
  \end{frame}


  % Frame 3 - Discuss Function parameters
\begin{frame}[fragile]
    \frametitle{Inferring function parameters}
    \begin{block}{}
    We use the restrictions on how the parameter is used in the function body to narrow the set of possible types the parameter can assume.
    \end{block}
    
    \begin{block}{}
    \begin{lstlisting}
        def f(x):
          for y in x:
            print(y)
          x + "Hello world"
    \end{lstlisting}
    \end{block}
\end{frame}
 
  
  % Frame 4 - Discuss Trys/Excepts
\begin{frame}[fragile]
    \frametitle{Inferring types in try/excepts}
    We assume that every statement within a \texttt{try} can pass control to exception blocks
    \begin{block}{}
    \begin{lstlisting}
        	try:
        	  x = 5 / possible_zero
        	  x = some_file.readline()
        	except IOError:
        	  y = x
    \end{lstlisting}
    \end{block}
\end{frame}

\begin{frame}[fragile]
    \frametitle{Global and local variables}
    \begin{itemize}
    \item \textbf{Local variables} - At a given point, a local variable is limited to the types of the immediately previous assignment(s)
    		\item \textbf{Global variables} - The type of a global variable is the union of every possible type it can assume in the program
    \end{itemize}
    
    \begin{lstlisting}
        def f():
          x = 5
          x = "Hi"
    \end{lstlisting}
    \begin{lstlisting}
        x = 5
        def f():
          global x
          x = "Hi"
          
        def g():
          y = x
    \end{lstlisting}
\end{frame}
  
  
  % Frame 4 - Demo
  \begin{frame}
    \frametitle{Demo}
    Thank you for listening \\
    We will now show our tool finding some type errors
  \end{frame}
  
  
  
  
  
  
  
  
  
\end{document}